%NO MODIFICAR ESTA SECCION!
\documentclass{article} % Define la clase del documento, en este caso, un artículo

\usepackage[letterpaper,margin=3cm]{geometry} % Configura el tamaño del papel y los márgenes del documento
\usepackage{graphicx} % Permite la inserción de imágenes
\usepackage[spanish]{babel}% Activar esta configuración para informes en español, ajusta el idioma del documento
\usepackage[usenames]{color} % Permite el uso de colores definidos por nombre en el documento
\usepackage{hyperref} % Habilita enlaces y referencias dentro del documento
\hypersetup{colorlinks=true, linkcolor = black, citecolor= black} % Configura el color de los enlaces y citas
\usepackage{booktabs} % Proporciona comandos para crear tablas de alta calidad
\usepackage{natbib} % Permite el uso de citas y referencias bibliográficas con diferentes estilos
\usepackage{tikz} % Permite la creación de gráficos y diagramas vectoriales directamente en LaTeX
\usepackage{float} % Para controlar la posición de los elementos flotantes, como imágenes, con la opción [H]
\bibliographystyle{agsm} % Define el estilo de citas y bibliografía (en este caso, el estilo AGSM)
\usepackage{diagbox} % Permite crear celdas con líneas diagonales en tablas
\usepackage{listings} % Permite la inclusión y formateo de código fuente en el documento
\usepackage{xcolor} % Paquete para definir y usar colores en el documento
\usepackage{parskip} % Añade espacio entre párrafos en lugar de sangrías
\usepackage{fancyhdr} % Permite personalizar encabezados y pies de página
\usepackage{amsmath} % Proporciona una amplia variedad de entornos y comandos matemáticos
\usepackage{enumitem}
\usepackage{tcolorbox}

% Definimos colores
\definecolor{levelone}{RGB}{230, 230, 250}   % lavanda claro
\definecolor{leveltwo}{RGB}{255, 228, 225}   % rosa claro
\definecolor{levelthree}{RGB}{224, 255, 255} % cian claro
\definecolor{levelfour}{RGB}{240, 255, 240}  % verde claro

\newtcolorbox{highlightbox}[1][]{colback=#1, colframe=white, boxrule=0pt, arc=0pt, left=2pt, right=2pt, top=2pt, bottom=2pt}

\pagestyle{fancy} % Usa el estilo fancyhdr
\fancyhf{} % Borra todos los encabezados y pies de página
\renewcommand{\headrulewidth}{0pt}
\renewcommand{\footrulewidth}{0pt} % Desactiva la línea horizontal predeterminada en el pie
\setlength{\headheight}{2cm} % Ajusta la altura del encabezado para hacer espacio para la línea
\fancyhead[L]{\raisebox{0.20cm}{\textbf{Métodos y técnicas de cosntrucción}}} % Añade el texto en la parte izquierda del encabezado, subiéndolo ligeramente
\fancyhead[R]{\raisebox{0.1cm}{\includegraphics[width=0.25\linewidth]{LOGO_UNIVERSIDAD.jpg}}} % Añade la imagen en la parte derecha del encabezado y súbela un poco
\fancyhead[C]{\rule{\textwidth}{0.6pt}} % Añade una línea horizontal superior centrada
\fancyfoot[C]{\rule{\textwidth}{0.6pt}} % Añade una línea horizontal en el pie de página centrada
\fancyfoot[R]{\raisebox{-1.5\baselineskip}{\thepage}} % Coloca el número de página a la derecha, con suficiente espacio debajo de la línea
\geometry{top=3cm, bottom=2.5cm} % Ajusta los márgenes superior e inferior

% Definición de colores al estilo Visual Studio Code
\definecolor{codegreen}{rgb}{0.25,0.49,0.48} % Comentarios
\definecolor{codegray}{rgb}{0.5,0.5,0.5} % Números y anotaciones
\definecolor{codepurple}{rgb}{0.58,0,0.82} % Palabras clave
\definecolor{backcolour}{rgb}{0.95,0.95,0.92} % Color de fondo

% Configuración del estilo de las celdas de código
\lstset{
    backgroundcolor=\color{backcolour},   % color de fondo; necesita que el paquete color o xcolor esté cargado
    commentstyle=\color{codegreen},       % estilo de comentarios
    keywordstyle=\color{codepurple},      % estilo de palabras clave
    numberstyle=\tiny\color{codegray},    % estilo de los números de línea
    stringstyle=\color{red},              % estilo de las cadenas de texto
    basicstyle=\ttfamily\small,           % estilo del texto básico
    breakatwhitespace=false,              % ajustes de líneas sólo en espacios en blanco
    breaklines=true,                      % ajustar las líneas si son muy largas
    captionpos=b,                         % posición de la leyenda (abajo)
    keepspaces=true,                      % preserva los espacios en el texto; útil si se usa monoespaciado
    numbers=left,                         % dónde poner los números de línea
    numbersep=5pt,                        % qué tan lejos están los números de línea del código
    showspaces=false,                     % mostrar espacios con subrayados particulares; reemplaza 'showstringspaces'
    showstringspaces=false,               % subrayar los espacios dentro de las cadenas solo
    showtabs=false,                       % mostrar tabulaciones en el código con subrayados particulares
    tabsize=2,                            % tamaños de tabulación a 2 espacios
    language=TeX,                         % lenguaje del código
    morecomment=[l]\#,                    % reconocer # como inicio de comentario en Python
    frame=single,                         % agregar un marco simple alrededor del código
    rulecolor=\color{black}               % color del marco
}

\begin{document}
%----------------------------------------------------------------------------------------
%   PORTADA
%Modificar desde aqui en adelante
%----------------------------------------------------------------------------------------
\begin{titlepage}%Inicio de la carátula, solo modificar los datos necesarios
\newcommand{\HRule}{\rule{\linewidth}{0.5mm}} 
\center 
%----------------------------------------------------------------------------------------
%	ENCABEZADO
%----------------------------------------------------------------------------------------
\includegraphics[width=10cm]{LOGO_UNIVERSIDAD.jpg}\\ % Si esta plantilla se copio correctamente, va a llevar la imagen del logo de la facultad.OBS: Es necesario incluir el paquete: graphicx
\vspace{3cm}
%----------------------------------------------------------------------------------------
%	SECCION DEL TITULO
%----------------------------------------------------------------------------------------
\HRule \\[0.4cm]
{ \huge \bfseries Métodos y Técnicas de construcción}\\[0.4cm] % Titulo del documento
{ \huge \bfseries Resumen prueba 1}\\[0.4cm] % Titulo del documento
\HRule \\[1.5cm]
 \vspace{5cm}
%----------------------------------------------------------------------------------------
%	SECCION DEL AUTOR
%----------------------------------------------------------------------------------------
\begin{flushright}
    { \textbf{Autores:}\\
    Bernardo Caprile Canala-Echevarría\\
    Felipe Vicencio\\
    Lukas Wolff\\
    }
\end{flushright}
\vspace{1cm}
%----------------------------------------------------------------------------------------
%	SECCION DE LA FECHA
%----------------------------------------------------------------------------------------
{\large \textbf{\today}}\\[2cm] % El comando \today coloca la fecha del dia, y esto se actualiza con cada compilacion, en caso de querer tener una fecha estatica, reemplazar el \today por la fecha deseada
\end{titlepage}
%----------------------------------------------------------------------------------------
%  INDICE
%----------------------------------------------------------------------------------------
\newpage
\tableofcontents
\thispagestyle{plain} % Deshabilita el encabezado en la página del índice
\thispagestyle{empty} % Deshabilita el número de página en la página del índice
\newpage

%Se puede agregar un indice de figuras si es nesesario
%\newpage
%\listoffigures 
%\thispagestyle{plain} % Deshabilita el encabezado en la página del índice %
%\thispagestyle{empty}
%\newpage
%----------------------------------------------------------------------------------------
%   ACÁ EMPIEZA EL INFORME
\setcounter{page}{1} % Reinicia el contador de páginas
%----------------------------------------------------------------------------------------
%Este es el formato a seguir para los titulos de las secciones
\section{Capitulo 1}

\subsection{Proyecto}

\begin{itemize}[label={},left=0pt,align=parleft]
    \item \begin{highlightbox}[levelone] Es un conjunto de actividades relacionadas entre sí \end{highlightbox}
    \item \begin{highlightbox}[levelone] Los proyectos son únicos \end{highlightbox}
    \item \begin{highlightbox}[levelone] Relaciona un equipo de trabajo, en un periodo de tiempo bajo requisitos específicos \end{highlightbox}
\end{itemize}

\subsection{Tipos de Proyectos}

\begin{itemize}[label={},left=0pt,align=parleft]
    \item \begin{highlightbox}[levelone] Proyectos de Construcción \end{highlightbox}
    \begin{itemize}[label={},left=1em,align=parleft]
        \item \begin{highlightbox}[leveltwo] Es un tipo de proyecto que tiene asignados objetivos, especificaciones, plazo y presupuesto \end{highlightbox}
        \item \begin{highlightbox}[leveltwo] Tipos de construcciones: \end{highlightbox}
        \begin{itemize}[label={},left=2em,align=parleft]
            \item \begin{highlightbox}[levelthree] Habitacional \end{highlightbox}
            \item \begin{highlightbox}[levelthree] No habitacional \end{highlightbox}
            \item \begin{highlightbox}[levelthree] Industrial \end{highlightbox}
            \item \begin{highlightbox}[levelthree] Obras Civiles \end{highlightbox}
        \end{itemize}
        \item \begin{highlightbox}[leveltwo] Tipos de Vida: \end{highlightbox}
        \begin{itemize}[label={},left=2em,align=parleft]
            \item \begin{highlightbox}[levelthree] Vida de Diseño: Es la vista prevista del proyecto, es la que se espera que tenga. \end{highlightbox}
            \item \begin{highlightbox}[levelthree] Vida Útil: Es la duración estimada que un objeto debe tener, respecto a factores externos. \end{highlightbox}
            \item \begin{highlightbox}[levelthree] Vida Remanente: Es el periodo durante el cual un objeto puede utilizarse de forma rentable antes de que la mantención ya no sea viable. \end{highlightbox}
        \end{itemize}
        \item \begin{highlightbox}[leveltwo] Etapas de un Proyecto de Construcción: \end{highlightbox}
        \begin{itemize}[label={},left=2em,align=parleft]
            \item \begin{highlightbox}[levelthree] Existe una necesidad \end{highlightbox}
            \item \begin{highlightbox}[levelthree] Análisis \end{highlightbox}
            \item \begin{highlightbox}[levelthree] Identificación de soluciones \end{highlightbox}
            \item \begin{highlightbox}[levelthree] Estudios de Factibilidad \end{highlightbox}
            \item \begin{highlightbox}[levelthree] Evaluación \end{highlightbox}
            \item \begin{highlightbox}[levelthree] Financiamiento \end{highlightbox}
            \item \begin{highlightbox}[levelthree] Diseño, que considera los siguientes aspectos: \end{highlightbox}
            \begin{itemize}[label={},left=3em,align=parleft]
                \item \begin{highlightbox}[levelfour] Estudio de Terreno \end{highlightbox}
                \item \begin{highlightbox}[levelfour] Diseño Arquitectónico \end{highlightbox}
                \item \begin{highlightbox}[levelfour] Diseño Estructural \end{highlightbox}
                \item \begin{highlightbox}[levelfour] Estudios de Impacto Ambiental \end{highlightbox}
                \item \begin{highlightbox}[levelfour] Diseño de Instalaciones \end{highlightbox}
                \item \begin{highlightbox}[levelfour] Redacción de documentos de licitación \end{highlightbox}
                \item \begin{highlightbox}[levelfour] Constructibilidad y Mantención \end{highlightbox}
            \end{itemize}
            \item \begin{highlightbox}[levelthree] Licitacion \end{highlightbox}
            \item \begin{highlightbox}[levelthree] Construcción \end{highlightbox}
            \item \begin{highlightbox}[levelthree] Puesta en Marcha \end{highlightbox}
        \end{itemize}
    \end{itemize}
\end{itemize}

De esta manera, un proyecto de construcción se puede expresar de la siguiente manera:

\begin{figure}[H]
    \centering
    \includegraphics[width=0.5\textwidth]{proyecto_construccion.png}
    \caption{Proyecto de Construcción}
    \label{fig:ProyectoConstruccion}
\end{figure}

\section{Capítulo 2}

\subsection{Diseño de un Proyecto de Construcción}

\begin{itemize}[label={},left=0pt,align=parleft]
    \item \begin{highlightbox}[levelone] Estudio de terreno, el cual consta de: \end{highlightbox}
    \begin{itemize}[label={},left=1em,align=parleft]
        \item \begin{highlightbox}[leveltwo] Ubicación del terreno \end{highlightbox}
        \item \begin{highlightbox}[leveltwo] Condiciones propias tales como: \end{highlightbox}
        \begin{itemize}[label={},left=2em,align=parleft]
            \item \begin{highlightbox}[levelthree] Topografía \end{highlightbox}
            \item \begin{highlightbox}[levelthree] Geología \end{highlightbox}
            \item \begin{highlightbox}[levelthree] Hidrología \end{highlightbox}
            \item \begin{highlightbox}[levelthree] Fuentes de Abastecimiento como energía y comunicaciones \end{highlightbox}
        \end{itemize}
        \item \begin{highlightbox}[leveltwo] Aspectos Legales, específicos a cada zona. \end{highlightbox}
        \item \begin{highlightbox}[leveltwo] Condiciones de servicio, como agua potable, electricidad o alcantarillado. \end{highlightbox}
        \item \begin{highlightbox}[leveltwo] Evaluación de impacto ambiental. \end{highlightbox}
    \end{itemize}
\end{itemize}

\subsection{Leyes}

\begin{itemize}[label={},left=0pt,align=parleft]
    \item \begin{highlightbox}[levelone] Ley general de urbanismo y construcciones (DFL 458, MINVU): Contiene el proceso global de urbanismo y construcción. \end{highlightbox}
    \item \begin{highlightbox}[levelone] Ley Base del Medio Ambiente (Ley 19.300): Regula el derecho a vivir en un medio ambiente libre de contaminación. \end{highlightbox}
    \item \begin{highlightbox}[levelone] Ley para la construcción de viviendas económicas (DFL-2 de 1959): Desarrollo el concepto de vivienda económica como aquella que tiene max 140 $m^2$ y no excede los 17.5 $m^2$ edificados por cama. \end{highlightbox}
    \item \begin{highlightbox}[levelone] Decreto Ley 2552-1979: Busca resolver los problemas de marginalidad habitacional, también define el concepto de vivienda de emergencia. \end{highlightbox}
    \item \begin{highlightbox}[levelone] Código del Trabajo (2002): Regula remuneraciones, gratificaciones, contratos, descansos, etc. \end{highlightbox}
    \item \begin{highlightbox}[levelone] Ley sobre accidente de trabajo y enfermedades profesionales (16.744): Establece un seguro obligatorio contra accidentes del trabajo y enfermedades profesionales. \end{highlightbox}
    \item \begin{highlightbox}[levelone] Ley de subcontratación (20.123): Regula la subcontratación de trabajadores. \end{highlightbox}
    \item \begin{highlightbox}[levelone] Ley de Concesiones (DFL 164) y Reglamento (DS 240) de Concesiones de Obras Públicas: Regula la concesión de obras públicas. \end{highlightbox}
    \item \begin{highlightbox}[levelone] Código Civil: El constructor tiene una responsabilidad de 5 años sobre la obra. \end{highlightbox}
    \item \begin{highlightbox}[levelone] Ley de la venta por piso o ley de propiedad horizontal (Ley 6071): Regula la venta de departamentos en construcción. \end{highlightbox}
    \item \begin{highlightbox}[levelone] Ley que incorpora el IVA a las empresas constructoras (Ley 18.630). \end{highlightbox}
    \item \begin{highlightbox}[levelone] etc. \end{highlightbox}
\end{itemize}

\subsection{Normas}

INN $=>$ Instituto Nacional de Normalización, cumplir sus normas no es de carácter obligatorio. Algunas de las áreas que cubre:

\begin{itemize}[label={},left=0pt,align=parleft]
    \item \begin{highlightbox}[levelone] General \end{highlightbox}
    \item \begin{highlightbox}[levelone] Diseño Arquitectónico \end{highlightbox}
    \item \begin{highlightbox}[levelone] Diseño, Cálculo y Ejecución de Estructuras \end{highlightbox}
    \item \begin{highlightbox}[levelone] Acondicionamiento Ambiental \end{highlightbox}
    \item \begin{highlightbox}[levelone] Materiales y Componentes \end{highlightbox}
    \item \begin{highlightbox}[levelone] Instalaciones \end{highlightbox}
    \item \begin{highlightbox}[levelone] Herramientas \end{highlightbox}
\end{itemize}

\subsection{Especificaciones Técnicas}

Corresponden a documentos asociados al proyecto, y sirven como complemento hacia los planos.

\subsection{Permisos y derechos de Construcción}

Las obras privadas deben tener un permiso de construcción, antes de comenzar su ejecución.

\subsection{Permisos de Construcción}

Se solicita a la dirección de obras municipales, para su obtención, se debe seguir el siguiente proceso:

\begin{itemize}[label={},left=0pt,align=parleft]
    \item \begin{highlightbox}[levelone] Solicitud de permiso: firmada por el propietario y arquitecto del proyecto \end{highlightbox}
    \item \begin{highlightbox}[levelone] Legado de documentos, que incluye: \end{highlightbox}
    \begin{itemize}[label={},left=1em,align=parleft]
        \item \begin{highlightbox}[leveltwo] Fotocopia de certificado y informaciones previas. \end{highlightbox}
        \item \begin{highlightbox}[leveltwo] Formulario único de estadísticas de edificación. \end{highlightbox}
        \item \begin{highlightbox}[leveltwo] Certificado de factibilidad de servicios \end{highlightbox}
        \item \begin{highlightbox}[leveltwo] Planos de Arquitectura \end{highlightbox}
        \item \begin{highlightbox}[leveltwo] Proyecto de cálculo estructural \end{highlightbox}
        \item \begin{highlightbox}[leveltwo] Cuadros de superficie \end{highlightbox}
        \item \begin{highlightbox}[leveltwo] Especificaciones técnicas de las partidas \end{highlightbox}
        \item \begin{highlightbox}[leveltwo] Levantamiento topográfico \end{highlightbox}
    \end{itemize}
    \item \begin{highlightbox}[levelone] Pago de derechos municipales \end{highlightbox}
    \item \begin{highlightbox}[levelone] Firma de documentos \end{highlightbox}
\end{itemize}

\subsection{Sistema de Evaluación de Impacto Ambiental SEIA}

Se establece que toda persona tiene derecho a vivir en un ambiente libre de contaminación. El SEIA es un instrumento de carácter preventivo, donde se determina si un proyecto se ajusta a las normas ambientales vigentes.

Los siguientes proyectos deben someterse a SEIA:

\begin{itemize}[label={},left=0pt,align=parleft]
    \item \begin{highlightbox}[levelone] Acueductos, embalses o tranques \end{highlightbox}
    \item \begin{highlightbox}[levelone] Líneas de transmisión eléctrica de alto voltaje \end{highlightbox}
    \item \begin{highlightbox}[levelone] Aeropuertos, terminales de buses, camiones y trenes \end{highlightbox}
    \item \begin{highlightbox}[levelone] Proyectos de desarrollo urbano y turístico \end{highlightbox}
    \item \begin{highlightbox}[levelone] Instalaciones fabriles \end{highlightbox}
    \item \begin{highlightbox}[levelone] Agroindustrias \end{highlightbox}
    \item \begin{highlightbox}[levelone] Proyectos de explotación forestal \end{highlightbox}
    \item \begin{highlightbox}[levelone] Proyectos que conllevan el uso de sustancias tóxicas \end{highlightbox}
    \item \begin{highlightbox}[levelone] Proyectos de saneamiento ambiental \end{highlightbox}
    \item \begin{highlightbox}[levelone] Ejecución de obras en parques nacionales \end{highlightbox}
\end{itemize}

Documentos que deben presentarse:

\begin{itemize}[label={},left=0pt,align=parleft]
    \item \begin{highlightbox}[levelone] Descripción del proyecto \end{highlightbox}
    \item \begin{highlightbox}[levelone] Un plan de cumplimiento de legislación vigente \end{highlightbox}
    \item \begin{highlightbox}[levelone] Razones que hacen necesaria el EIA y no DIA \end{highlightbox}
    \item \begin{highlightbox}[levelone] Condiciones ambientales previas al proyecto \end{highlightbox}
    \item \begin{highlightbox}[levelone] Predicción de los impactos ambientales por el proyecto \end{highlightbox}
    \item \begin{highlightbox}[levelone] Medidas que se tomarán para eliminar o disminuir los impactos \end{highlightbox}
    \item \begin{highlightbox}[levelone] Acciones previas al estudio \end{highlightbox}
\end{itemize}

\subsubsection{Estudio de Impacto Ambiental (EIA)}

Es un conjunto de estudios necesarios para evaluar el impacto ambiental de un proyecto. Se debe presentar un EIA cuando un proyecto presenta uno de los siguientes impactos:

\begin{itemize}[label={},left=0pt,align=parleft]
    \item \begin{highlightbox}[levelone] Riesgosa para la salud de la población \end{highlightbox}
    \item \begin{highlightbox}[levelone] Efectos adversos sobre recursos naturales renovables \end{highlightbox}
    \item \begin{highlightbox}[levelone] Alteración de comunidades en los sistemas de vida \end{highlightbox}
    \item \begin{highlightbox}[levelone] Localización próxima a poblaciones \end{highlightbox}
    \item \begin{highlightbox}[levelone] Recursos o áreas protegidas \end{highlightbox}
    \item \begin{highlightbox}[levelone] Alteración significativa del valor paisajístico y/o turístico de la zona \end{highlightbox}
    \item \begin{highlightbox}[levelone] Alteración de monumentos \end{highlightbox}
\end{itemize}

\subsubsection{Declaración de Impacto Ambiental (DIA)}

Si un proyecto no presenta alguno de los impactos anteriores, solo debe presentar una declaración de impacto ambiental (DIA). Debe explicar por qué no es necesario el EIA, declarando sus compromisos ambientales. \\ \\
La principal diferencia entre un EIA y una DIA es que el EIA es un estudio más profundo y detallado, mientras que la DIA es un estudio más superficial. \\ \\
El DIA se rechaza si:

\begin{itemize}[label={},left=0pt,align=parleft]
    \item \begin{highlightbox}[levelone] No cumple la normativa \end{highlightbox}
    \item \begin{highlightbox}[levelone] No se subsanan los errores, omisiones o inexactitudes de ella \end{highlightbox}
    \item \begin{highlightbox}[levelone] El respectivo proyecto o actividad requiere de un EIA \end{highlightbox}
\end{itemize}

\subsection{Participantes de un proyecto de construcción}

\begin{itemize}[label={},left=0pt,align=parleft]
    \item \begin{highlightbox}[levelone] Durante el estudio y diseño: \end{highlightbox}
    \begin{itemize}[label={},left=1em,align=parleft]
        \item \begin{highlightbox}[leveltwo] Consultores Financieros \end{highlightbox}
        \item \begin{highlightbox}[leveltwo] Arquitectos \end{highlightbox}
        \item \begin{highlightbox}[leveltwo] Ingenieros \end{highlightbox}
        \item \begin{highlightbox}[leveltwo] Asesores Legales, Ambientales y de Construcción \end{highlightbox}
        \item \begin{highlightbox}[leveltwo] Otros \end{highlightbox}
    \end{itemize}
    \item \begin{highlightbox}[levelone] Durante la construcción: \end{highlightbox}
    \begin{itemize}[label={},left=1em,align=parleft]
        \item \begin{highlightbox}[leveltwo] Empresas constructoras \end{highlightbox}
        \item \begin{highlightbox}[leveltwo] Subcontratistas \end{highlightbox}
        \item \begin{highlightbox}[leveltwo] Inspección técnica de la obra (ITO) \end{highlightbox}
        \item \begin{highlightbox}[leveltwo] Organismos reguladores \end{highlightbox}
        \item \begin{highlightbox}[leveltwo] Proveedores \end{highlightbox}
        \item \begin{highlightbox}[leveltwo] Laboratorios de control de calidad \end{highlightbox}
        \item \begin{highlightbox}[leveltwo] Abogados \end{highlightbox}
        \item \begin{highlightbox}[leveltwo] Entidades de seguros \end{highlightbox}
        \item \begin{highlightbox}[leveltwo] Entidades ambientales \end{highlightbox}
        \item \begin{highlightbox}[leveltwo] Auditores bancarios \end{highlightbox}
        \item \begin{highlightbox}[leveltwo] Visitadores de obras \end{highlightbox}
        \item \begin{highlightbox}[leveltwo] Otros \end{highlightbox}
    \end{itemize}
\end{itemize}

\newpage 
\section{Capítulo 3}

\subsection{Elementos de la gestión de la construcción}

\begin{minipage}{0.45\textwidth}
    \begin{itemize}[label={},left=0pt,align=parleft]
        \item \begin{highlightbox}[levelone] Hay que encontrar un balance entre estos 3 ítems \end{highlightbox}
        \item \begin{highlightbox}[levelone] Una buena gestión de Calidad reducirá los plazos \end{highlightbox}
        \item \begin{highlightbox}[levelone] Una buena gestión de calidad mantendrá o reducirá los costos \end{highlightbox}
    \end{itemize}
\end{minipage}
\hfill
\begin{minipage}{0.45\textwidth}
    \centering
    \includegraphics[width=\textwidth]{gestion_recursos.png}
\end{minipage}

\subsection{Factibilidad de un Proyecto de Construcción}

\begin{minipage}{0.45\textwidth}
    \begin{itemize}[label={},left=0pt,align=parleft]
        \item \begin{highlightbox}[levelone] No es factible si al planificarlo existe la posibilidad de que se sobrepasen límites de plazos o presupuestos \end{highlightbox}
        \item \begin{highlightbox}[levelone] No es factible si no se cuenta con equipos o mano de obra o materiales que garanticen la calidad del proyecto \end{highlightbox}
    \end{itemize}
\end{minipage}
\hfill
\begin{minipage}{0.5\textwidth}
    \centering
    \includegraphics[width=1.2\textwidth]{factibilidad_proyecto.png}
\end{minipage}

\begin{minipage}{0.45\textwidth}
    \begin{itemize}[label={},left=0pt,align=parleft]
        \item \begin{highlightbox}[levelone] El proyecto se materializará a través de varias etapas: \end{highlightbox}
        \begin{itemize}[label={},left=1em,align=parleft]
            \item \begin{highlightbox}[leveltwo] Estudio y desarrollo del proyecto de ingeniería definitivo \end{highlightbox}
            \item \begin{highlightbox}[leveltwo] Construcción y puesta en marcha de la obra \end{highlightbox}
        \end{itemize}
        \item \begin{highlightbox}[levelone] La incertidumbre del costo depende de que tan bien se desarrollen las etapas \end{highlightbox}
    \end{itemize}
\end{minipage}
\hfill
\begin{minipage}{0.5\textwidth}
    \centering
    \includegraphics[width=1.2\textwidth]{etapas_proyecto.png}
\end{minipage}

\subsection{Administración de Proyectos}

\begin{minipage}{0.45\textwidth}
    \begin{itemize}[label={},left=0pt,align=parleft]
        \item \begin{highlightbox}[levelone] La administración tiene por objetivo transformar una decisión de inversión en una obra física \end{highlightbox}
        \item \begin{highlightbox}[levelone] Especificaciones de los objetivos del proyecto \end{highlightbox}
        \item \begin{highlightbox}[levelone] Maximización de los recursos \end{highlightbox}
        \item \begin{highlightbox}[levelone] Coordinación \end{highlightbox}
        \item \begin{highlightbox}[levelone] Comunicación efectiva \end{highlightbox}
    \end{itemize}
\end{minipage}
\hfill
\begin{minipage}{0.5\textwidth}
    \centering
    \includegraphics[width=1.2\textwidth]{piramide_administracion.png}
\end{minipage}

\subsection{Mandante y demás participantes}

Ventajas de realizar obras por medio de empresas contratistas:
\begin{itemize}[label={},left=0pt,align=parleft]
    \item \begin{highlightbox}[levelone] Se ocupan recursos externos, que son regulados con contratos \end{highlightbox}
    \item \begin{highlightbox}[levelone] La empresa tiene personal especializado y experiencia en la construcción \end{highlightbox}
    \item \begin{highlightbox}[levelone] La empresa se organiza distribuyendo en las variadas faenas su personal (No los despiden, los trasladan) \end{highlightbox}
    \item \begin{highlightbox}[levelone] Las empresas son más eficientes en: \end{highlightbox}
    \begin{itemize}[label={},left=1em,align=parleft]
        \item \begin{highlightbox}[leveltwo] Recursos físicos \end{highlightbox}
        \item \begin{highlightbox}[leveltwo] Inversión \end{highlightbox}
        \item \begin{highlightbox}[leveltwo] Adquisición de materiales \end{highlightbox}
    \end{itemize}
\end{itemize}

\newpage

\subsection{Planificación y control de proyectos de obras de construcción}

\begin{itemize}[label={},left=0pt,align=parleft]
    \item \begin{highlightbox}[levelone] Racionalizar las actividades del proceso constructivo \end{highlightbox}
    \item \begin{highlightbox}[levelone] Racionalizar los recursos y establecer control de estos \end{highlightbox}
    \item \begin{highlightbox}[levelone] Beneficios: \end{highlightbox}
    \begin{itemize}[label={},left=1em,align=parleft]
        \item \begin{highlightbox}[leveltwo] Reducir incertidumbre \end{highlightbox}
        \item \begin{highlightbox}[leveltwo] Conocer los volúmenes peak \end{highlightbox}
        \item \begin{highlightbox}[leveltwo] Programar los movimientos de la instalación y retiro de faena \end{highlightbox}
        \item \begin{highlightbox}[leveltwo] Optimizar la programación de área de la obra \end{highlightbox}
        \item \begin{highlightbox}[leveltwo] Elaborar un programa de adquisiciones de materiales y arriendo de equipos \end{highlightbox}
        \item \begin{highlightbox}[leveltwo] Definir períodos de contratos y despidos \end{highlightbox}
        \item \begin{highlightbox}[leveltwo] Establecer metodologías de control \end{highlightbox}
    \end{itemize}
\end{itemize}

\subsubsection{Niveles de planificación}

\begin{itemize}[label={},left=0pt,align=parleft]
    \item \begin{highlightbox}[levelone] \textbf{Planificación estratégica} \end{highlightbox}
    \begin{itemize}[label={},left=1em,align=parleft]
        \item \begin{highlightbox}[leveltwo] Se realiza a largo plazo \end{highlightbox}
        \item \begin{highlightbox}[leveltwo] Se establecen los aspectos globales del proyecto \end{highlightbox}
        \item \begin{highlightbox}[leveltwo] Se determinan costos estimados para propuestas \end{highlightbox}
    \end{itemize}
    \item \begin{highlightbox}[levelone] \textbf{Planificación táctica} \end{highlightbox}
    \begin{itemize}[label={},left=1em,align=parleft]
        \item \begin{highlightbox}[leveltwo] Se realiza a mediano plazo \end{highlightbox}
        \item \begin{highlightbox}[leveltwo] Planifica la materialización del proyecto \end{highlightbox}
        \item \begin{highlightbox}[leveltwo] Plan de construcción de general a detalle \end{highlightbox}
    \end{itemize}
    \item \begin{highlightbox}[levelone] \textbf{Planificación operativa} \end{highlightbox}
    \begin{itemize}[label={},left=1em,align=parleft]
        \item \begin{highlightbox}[leveltwo] Se realiza a corto plazo \end{highlightbox}
        \item \begin{highlightbox}[leveltwo] Cómo ejecutar las tareas necesarias para materializar las etapas del proyecto \end{highlightbox}
        \item \begin{highlightbox}[leveltwo] Nivel de detalle: planificación semanal o incluso diaria \end{highlightbox}
    \end{itemize}
\end{itemize}

\subsection{Gestión y control de costos de proyectos}

Incluye los procesos necesarios para asegurar que el proyecto se finalice dentro del presupuesto aprobado.
\begin{itemize}[label={},left=0pt,align=parleft]
    \item \begin{highlightbox}[levelone] Planificación de recursos \end{highlightbox}
    \item \begin{highlightbox}[levelone] Estimación de costos \end{highlightbox}
    \item \begin{highlightbox}[levelone] Presupuesto de costos \end{highlightbox}
    \item \begin{highlightbox}[levelone] Control de costos \end{highlightbox}
    \item \begin{highlightbox}[levelone] Costos del ciclo de vida del proyecto \end{highlightbox}
    \begin{itemize}[label={},left=1em,align=parleft]
        \item \begin{highlightbox}[leveltwo] Limitación de revisiones del diseño disminuye costos del proyecto, pero aumenta los costos de operación y mantención del mandante \end{highlightbox}
    \end{itemize}
\end{itemize}

\subsection{Control de costos}

\begin{minipage}{0.45\textwidth}
    \begin{itemize}[label={},left=0pt,align=parleft]
        \item \begin{highlightbox}[levelone] Influencia en factores que causan cambios en la base de costos \end{highlightbox}
        \item \begin{highlightbox}[levelone] Control del desarrollo de los costos \end{highlightbox}
        \item \begin{highlightbox}[levelone] Garantía de reflejar cambios apropiados en la base de costos \end{highlightbox}
        \item \begin{highlightbox}[levelone] Prevención de cambios incorrectos o inapropiados \end{highlightbox}
        \item \begin{highlightbox}[levelone] Comunicación de cambios autorizados a entidades involucradas \end{highlightbox}
    \end{itemize}
\end{minipage}
\hfill
\begin{minipage}{0.5\textwidth}
    \centering
    \includegraphics[width=\textwidth]{curva_s.png}
\end{minipage}

\begin{itemize}[label={},left=0pt,align=parleft]
    \item \begin{highlightbox}[levelone] Buscar causas de variaciones de costos (positivas y negativas). \end{highlightbox}
    \item \begin{highlightbox}[levelone] Integrar control de costos con otros procesos. \end{highlightbox}
    \item \begin{highlightbox}[levelone] Evitar respuestas inadecuadas que afecten calidad, programa o riesgos. \end{highlightbox}
    \item \begin{highlightbox}[levelone] Requiere: \end{highlightbox}
    \begin{itemize}[label={},left=1em,align=parleft]
        \item \begin{highlightbox}[leveltwo] Base de costos (resultado de la etapa anterior). \end{highlightbox}
        \item \begin{highlightbox}[leveltwo] Informes de realización (costos reales y cumplimiento de presupuestos). \end{highlightbox}
    \end{itemize}
    \item \begin{highlightbox}[levelone] Incluir cambios en el plan de gestión de costos. \end{highlightbox}
    \item \begin{highlightbox}[levelone] Herramientas y técnicas: \end{highlightbox}
    \begin{itemize}[label={},left=1em,align=parleft]
        \item \begin{highlightbox}[leveltwo] Control de cambios de costo. \end{highlightbox}
        \item \begin{highlightbox}[leveltwo] Medida de realización. \end{highlightbox}
        \item \begin{highlightbox}[leveltwo] Planificación adicional. \end{highlightbox}
        \item \begin{highlightbox}[leveltwo] Herramientas computarizadas. \end{highlightbox}
    \end{itemize}
\end{itemize}

\begin{itemize}[label={},left=0pt,align=parleft]
    \item \begin{highlightbox}[levelone] Estimaciones revisadas: Modificaciones de los costos del proyecto. \end{highlightbox}
    \item \begin{highlightbox}[levelone] Actualización del presupuesto: Cambios en la base de costos aprobada. \end{highlightbox}
    \item \begin{highlightbox}[levelone] Acciones correctivas: Ajustes para alinear el proyecto con el plan. \end{highlightbox}
    \item \begin{highlightbox}[levelone] Estimación al término: Previsión de los costos totales del proyecto: \end{highlightbox}
    \begin{itemize}[label={},left=1em,align=parleft]
        \item \begin{highlightbox}[leveltwo] Costos reales + presupuesto modificado por factor. \end{highlightbox}
        \item \begin{highlightbox}[leveltwo] Costos reales + nueva estimación para el trabajo pendiente. \end{highlightbox}
        \item \begin{highlightbox}[leveltwo] Costos reales + presupuesto restante. \end{highlightbox}
    \end{itemize}
    \item \begin{highlightbox}[levelone] Lecciones aprendidas: Causas de desviaciones y acciones correctoras. \end{highlightbox}
\end{itemize}

\subsection{Seguridad e higiene industrial}

\begin{itemize}[label={},left=0pt,align=parleft]
    \item \begin{highlightbox}[levelone] Motivos para preocuparse por la seguridad: \end{highlightbox}
    \begin{itemize}[label={},left=1em,align=parleft]
        \item \begin{highlightbox}[leveltwo] Responsabilidad de asegurar condiciones de trabajo seguras. \end{highlightbox}
        \item \begin{highlightbox}[leveltwo] Interés económico. \end{highlightbox}
        \item \begin{highlightbox}[leveltwo] Mandato legal. \end{highlightbox}
        \item \begin{highlightbox}[leveltwo] Imagen de la empresa. \end{highlightbox}
    \end{itemize}
    \item \begin{highlightbox}[levelone] La Seguridad Industrial busca realizar trabajos sin causar daño a personas ni afectar equipos, herramientas o áreas de trabajo. \end{highlightbox}
    \item \begin{highlightbox}[levelone] Los accidentes pueden y deben prevenirse, y es un deber moral y profesional. \end{highlightbox}
    \item \begin{highlightbox}[levelone] Un accidente es un hecho imprevisto que interfiere con el trabajo y puede causar: \end{highlightbox}
    \begin{itemize}[label={},left=1em,align=parleft]
        \item \begin{highlightbox}[leveltwo] Lesiones a personas. \end{highlightbox}
        \item \begin{highlightbox}[leveltwo] Daños a materiales, equipos, o propiedad. \end{highlightbox}
        \item \begin{highlightbox}[leveltwo] Interrupción del proceso productivo. \end{highlightbox}
    \end{itemize}
\end{itemize}

\begin{itemize}[label={},left=0pt,align=parleft]
    \item \begin{highlightbox}[levelone] No puede haber lesiones sin un accidente. \end{highlightbox}
    \item \begin{highlightbox}[levelone] Las causas de los accidentes son hechos o circunstancias que los provocan. \end{highlightbox}
    \item \begin{highlightbox}[levelone] Causas: \end{highlightbox}
    \begin{itemize}[label={},left=1em,align=parleft]
        \item \begin{highlightbox}[leveltwo] Acción insegura (85\%): Falta de protección, mal uso de herramientas, etc. \end{highlightbox}
        \item \begin{highlightbox}[leveltwo] Condición insegura (15\%): Problemas en el entorno, como cables sin aislación. \end{highlightbox}
    \end{itemize}
    \item \begin{highlightbox}[levelone] Los accidentes no tienen una causa única, sino que son el resultado de una cadena de circunstancias o eventos. \end{highlightbox}
\end{itemize}

\subsection{Costos y repercusiones socioeconómicas}

\begin{itemize}[label={},left=0pt,align=parleft]
    \item \begin{highlightbox}[levelone] Los accidentes generan costos directos e indirectos. \end{highlightbox}
    \item \begin{highlightbox}[levelone] Cotización básica: 0,95\% más cotización adicional según siniestralidad. \end{highlightbox}
    \item \begin{highlightbox}[levelone] Empresas nuevas: tasa presunta por dos años. \end{highlightbox}
    \item \begin{highlightbox}[levelone] Tasa de siniestralidad total incluye incapacidad temporal, invalidez y muertes. \end{highlightbox}
    \item \begin{highlightbox}[levelone] Siniestralidad aumenta si hay muertes por falta de prevención. \end{highlightbox}
    \item \begin{highlightbox}[levelone] Rebajas aplicables si se reducen accidentes. \end{highlightbox}
    \item \begin{highlightbox}[levelone] Costos indirectos son 4 veces mayores que los directos. \end{highlightbox}
    \item \begin{highlightbox}[levelone] Empleador financia seguro de accidentes (Ley 16.744). \end{highlightbox}
\end{itemize}

\subsection{Métodos generales de prevención de riesgos}

\begin{itemize}[label={},left=0pt,align=parleft]
    \item \begin{highlightbox}[levelone] Normas oficiales sobre seguridad y trabajo. \end{highlightbox}
    \item \begin{highlightbox}[levelone] Investigación estadística de accidentes. \end{highlightbox}
    \item \begin{highlightbox}[levelone] Normas de condiciones y obligaciones laborales. \end{highlightbox}
    \item \begin{highlightbox}[levelone] Inspección para cumplimiento de normas. \end{highlightbox}
    \item \begin{highlightbox}[levelone] Formación en seguridad. \end{highlightbox}
    \item \begin{highlightbox}[levelone] Conciencia de seguridad con afiches y charlas. \end{highlightbox}
    \item \begin{highlightbox}[levelone] Medidas de ingeniería para eliminar riesgos. \end{highlightbox}
    \item \begin{highlightbox}[levelone] Reacción ante accidentes y primeros auxilios. \end{highlightbox}
    \item \begin{highlightbox}[levelone] Investigación de accidentes. \end{highlightbox}
\end{itemize}

\subsection{Higiene Industrial}

\begin{itemize}[label={},left=0pt,align=parleft]
    \item \begin{highlightbox}[levelone] Las condiciones de trabajo o sustancias pueden causar daño o enfermedad profesional. \end{highlightbox}
    \item \begin{highlightbox}[levelone] La higiene industrial reconoce, evalúa y controla los riesgos de enfermedades profesionales. \end{highlightbox}
    \item \begin{highlightbox}[levelone] El objetivo es proteger la salud de los trabajadores a lo largo de su vida laboral. \end{highlightbox}
\end{itemize}

\newpage

\begin{table}[h!]
    \centering
    \begin{tabular}{|l|l|}
    \hline
    \textbf{Condiciones de trabajo}          & \textbf{Criterios de evaluación}                                  \\ \hline
    Accidentes laborales                     & - Índice de frecuencia                                            \\
                                             & - Índice de gravedad                                              \\
                                             & - Evaluación subjetiva de los riesgos                             \\ \hline
    Ambiente sonoro                          & - Medida del ambiente sonoro                                      \\
                                             & - Evaluación subjetiva del ambiente sonoro                        \\ \hline
    Vibraciones                              & - Medida de las vibraciones                                       \\
                                             & - Evaluación subjetiva de las vibraciones                         \\ \hline
    Ambiente térmico                         & - Características inmediatas del ambiente térmico                 \\
                                             & - Evaluación del confort térmico                                  \\
                                             & - Evaluación subjetiva del confort térmico                        \\ \hline
    Ambiente luminoso                        & - Características inmediatas del ambiente luminoso                \\
                                             & - Escala de confort visual                                        \\
                                             & - Evaluación subjetiva del confort visual                         \\ \hline
    \end{tabular}
    \caption{Condiciones de trabajo y criterios de evaluación (1)}
    \end{table}
    
    \begin{table}[h!]
    \centering
    \begin{tabular}{|l|l|}
    \hline
    \textbf{Condiciones de trabajo}          & \textbf{Criterios de evaluación}                                  \\ \hline
    Polución atmosférica (irritantes y tóxicos) & - Evaluación de la polución atmosférica                           \\
                                             & - Evaluación por sus efectos biológicos                           \\ \hline
    Trabajo nocturno y trabajo continuo equipos alternos & - Trabajo nocturno regular                             \\
                                             & - Trabajo continuo en equipos alternos                           \\ \hline
    Fatiga relacionada con el trabajo muscular dinámico y estático & - Definición de la postura de trabajo      \\
                                             & - Evaluación del consumo en energía                               \\
                                             & - Evaluación subjetiva del esfuerzo físico                        \\ \hline
    Carga y fatiga mental                    & - Evaluación subjetiva del grado de atención                      \\
                                             & - Evaluación subjetiva del grado de reflexión                     \\
                                             & - Evaluación subjetiva de la fatiga mental                        \\ \hline
    Ritmo de trabajo y autonomía temporal    & - Naturaleza del ritmo de trabajo                                 \\
                                             & - Pausas oficiales                                                \\
                                             & - Pausas autodecididas                                            \\
                                             & - Índices de pausas                                               \\
                                             & - Evaluación subjetiva del ritmo de trabajo                       \\
                                             & - Evaluación subjetiva del descanso                               \\ \hline
    \end{tabular}
    \caption{Condiciones de trabajo y criterios de evaluación (2)}
    \end{table}
    
    \begin{table}[h!]
    \centering
    \begin{tabular}{|l|l|}
    \hline
    \textbf{Condiciones de trabajo}          & \textbf{Criterios de evaluación}                                  \\ \hline
    Interés del trabajo y autonomía de decisión & - Naturaleza y número de puestos desempeñados                     \\
                                             & - Naturaleza y número de las actividades                          \\
                                             & - Duración del ciclo operativo                                    \\
                                             & - Afectación de las decisiones                                    \\
                                             & - Evaluación subjetiva del interés del trabajo                    \\
                                             & - Evaluación subjetiva de autonomía de decisión                   \\ \hline
    Relaciones en el centro de trabajo       & - Posibilidades de relaciones                                     \\
                                             & - Calidad de relaciones                                           \\
                                             & - Información sobre el futuro a corto plazo                       \\ \hline
    Conjunto de condiciones de trabajo       & - Indicadores de producción                                       \\
                                             & - Indicadores sociales                                            \\ \hline
    \end{tabular}
    \caption{Condiciones de trabajo y criterios de evaluación (3)}
    \end{table}

\newpage    


\subsection{Estadística y medición de accidentes}

\begin{itemize}[label={},left=0pt,align=parleft]
    \item \begin{highlightbox}[levelone] \textbf{Índice de Frecuencia (IF):} Cantidad de accidentes por millón de horas trabajadas. \end{highlightbox}
    \[
    IF = \frac{A \cdot 1,000,000}{B}
    \]
    Donde:
    \begin{itemize}[label={},left=1em,align=parleft]
        \item \begin{highlightbox}[leveltwo] $A$: Número de accidentes incapacitantes. \end{highlightbox}
        \item \begin{highlightbox}[leveltwo] $B$: Horas Hombre Trabajadas. \end{highlightbox}
    \end{itemize}
    
    \item \begin{highlightbox}[levelone] \textbf{Índice de Gravedad (IG):} Días de trabajo perdidos por accidentes, por cada millón de HH. \end{highlightbox}
    \[
    IG = \frac{A \cdot 1,000,000}{B}
    \]
    Donde:
    \begin{itemize}[label={},left=1em,align=parleft]
        \item \begin{highlightbox}[leveltwo] $A$: Días perdidos. \end{highlightbox}
        \item \begin{highlightbox}[leveltwo] $B$: Horas Hombre Trabajadas. \end{highlightbox}
    \end{itemize}
    
    \item \begin{highlightbox}[levelone] \textbf{Índice de Accidentes (IA):} Porcentaje de trabajadores que han sufrido un accidente en un período determinado. \end{highlightbox}
    \[
    IA = \frac{C \cdot 100}{D}
    \]
    Donde:
    \begin{itemize}[label={},left=1em,align=parleft]
        \item \begin{highlightbox}[leveltwo] $C$: Número de accidentes. \end{highlightbox}
        \item \begin{highlightbox}[leveltwo] $D$: Número de trabajadores. \end{highlightbox}
    \end{itemize}
\end{itemize}

\subsection{Seguridad e higiene industrial y la OHSAS 18001}

\begin{itemize}[label={},left=0pt,align=parleft]
    \item \begin{highlightbox}[levelone] Norma de gestión de seguridad e higiene industrial. \end{highlightbox}
    \item \begin{highlightbox}[levelone] Proporciona los elementos estructurales para un Sistema de Gestión de Seguridad y Salud Ocupacional. \end{highlightbox}
    \item \begin{highlightbox}[levelone] Objetivo: Revisar, gestionar y mejorar el control de riesgos laborales. \end{highlightbox}
\end{itemize}

\subsection{Sistemas de gestión de la calidad, norma ISO 9000}

\begin{itemize}[label={},left=0pt,align=parleft]
    \item \begin{highlightbox}[levelone] Un Sistema de Gestión de la Calidad asegura que una organización identifica y satisface las necesidades de sus clientes. \end{highlightbox}
    \item \begin{highlightbox}[levelone] Se enfoca en planificar, mantener y mejorar los procesos para lograr ventajas competitivas. \end{highlightbox}
    \item \begin{highlightbox}[levelone] Es una gestión centrada en la calidad, basada en la participación de todos los miembros de la organización. \end{highlightbox}
    \item \begin{highlightbox}[levelone] El objetivo principal es la satisfacción del cliente a través de la calidad del servicio. \end{highlightbox}
    \item \begin{highlightbox}[levelone] Para tener éxito, el servicio debe: \end{highlightbox}
    \begin{itemize}[label={},left=1em,align=parleft]
        \item \begin{highlightbox}[leveltwo] Satisfacer un propósito bien definido. \end{highlightbox}
        \item \begin{highlightbox}[leveltwo] Cumplir las expectativas del cliente. \end{highlightbox}
        \item \begin{highlightbox}[leveltwo] Cumplir con los requisitos legales y normas aplicables. \end{highlightbox}
    \end{itemize}
    \item \begin{highlightbox}[levelone] Un Sistema de Gestión de la Calidad incluye auditorías externas por parte de una empresa certificadora. \end{highlightbox}
    \item \begin{highlightbox}[levelone] Elementos básicos del sistema: \end{highlightbox}
    \begin{itemize}[label={},left=1em,align=parleft]
        \item \begin{highlightbox}[leveltwo] Estrategia \end{highlightbox}
        \item \begin{highlightbox}[leveltwo] Estructura \end{highlightbox}
        \item \begin{highlightbox}[leveltwo] Relaciones \end{highlightbox}
        \item \begin{highlightbox}[leveltwo] Recursos \end{highlightbox}
        \item \begin{highlightbox}[leveltwo] Documentación \end{highlightbox}
    \end{itemize}
    \item \begin{highlightbox}[levelone] La ISO (Organización Internacional de Normalización) fue creada en 1947, con sede en Ginebra. \end{highlightbox}
    \item \begin{highlightbox}[levelone] Las normas ISO 9000 del año 2000 incluyen: \end{highlightbox}
    \begin{itemize}[label={},left=1em,align=parleft]
        \item \begin{highlightbox}[leveltwo] ISO 9000:2000: Fundamentos y vocabulario. \end{highlightbox}
        \item \begin{highlightbox}[leveltwo] ISO 9001:2000: Requisitos de Sistemas de Gestión de Calidad. \end{highlightbox}
        \item \begin{highlightbox}[leveltwo] ISO 9004: Sistemas de Gestión de la Calidad. \end{highlightbox}
        \item \begin{highlightbox}[leveltwo] ISO 19011: Auditoría de sistemas de gestión de calidad y/o ambiental. \end{highlightbox}
    \end{itemize}
\end{itemize}



\newpage
\section{Capítulo 4: Contratos y propuestas en proyectos de Construcción}

Es un convenio entre el que construye y el dueño o mandante que financia, y fija sus objetivos
de acuerdo con sus necesidades y posibilidades. El propósito de un contrato de construcción
es definir derechos, obligaciones y responsabilidades de cada una de las partes involucradas.
\\ \\
El propietario puede designar una inspección para controlar la obra. Es como intermediario entre
contratista y mandante. Hay contratos que le adjutican esta responsabilidad a un "Árbitro".

\begin{itemize}[label={},left=0pt,align=parleft]
    \item \begin{highlightbox}[levelone] Modalidades de Contratos de construcción \end{highlightbox}
    \begin{itemize}[label={},left=1em,align=parleft]
        \item \begin{highlightbox}[leveltwo] Construir para sí: Usa sus propios recursos para la ejecución de proyectos arriesgándose a la no aceptación de este por parte del mercado inmobiliario. \end{highlightbox}
        \item \begin{highlightbox}[leveltwo] Construir por terceros: Obra es financiada por el mandante, y el contratista ejecuta. Pueden tener las siguientes relaciones. \end{highlightbox}
        \begin{itemize}[label={},left=2em,align=parleft]
            \item \begin{highlightbox}[levelthree] Mandante entrega el proyecto y lo financia, contratista lo ejecuta. \end{highlightbox}
            \item \begin{highlightbox}[levelthree] Mandante financia la obra, solicita el diseño y la ejecución al contratista. \end{highlightbox}
            \item \begin{highlightbox}[levelthree] Contratista diseña, ejecuta y financia la obra y entrega la obra terminada al mandante en un precio previamente convenido (contrato llave en mano). \end{highlightbox}
        \end{itemize}
    \end{itemize}
    \item \begin{highlightbox}[levelone] Tipos de Contratos \end{highlightbox}
    \begin{itemize}[label={},left=1em,align=parleft]
        \item \begin{highlightbox}[leveltwo] Contrato de suma alzada: Contratista realiza toda la obra a un precio fijo (propuesto por él después de estudiar el proyecto y aceptado por el mandante). Máximo riesgo es del contratista. \end{highlightbox}
        \begin{itemize}[label={},left=2em,align=parleft]
            \item \begin{highlightbox}[levelthree] Proyecto tiene que estar 100\% definido. \end{highlightbox}
            \item \begin{highlightbox}[levelthree] Dueño escoge la mejor oferta. \end{highlightbox}
            \item \begin{highlightbox}[levelthree] Los cambios son casi imposibles por parte del mandante, debido al contrato de adjudicación. \end{highlightbox}
            \item \begin{highlightbox}[levelthree] Contratista debe hacer un análisis de costos precisos para dar la oferta. \end{highlightbox}
        \end{itemize}
        \item \begin{highlightbox}[leveltwo] Contrato de precios unitarios: Se establecen precios unitarios para cada partida de la obra, y se paga por la cantidad de trabajo realizado. Se paga por lo que se hace. El riesgo es compartido entre mandante y contratista. Es competitivo. \end{highlightbox}
        \begin{itemize}[label={},left=2em,align=parleft]
            \item \begin{highlightbox}[levelthree] Se puede ofertar sin tener el proyecto definido. \end{highlightbox}
            \item \begin{highlightbox}[levelthree] Permite al dueño saber exactamente cuánto va a invertir en la obra. \end{highlightbox}
            \item \begin{highlightbox}[levelthree] Contratista deberá realizar un estudio de costos preciso. \end{highlightbox}
        \end{itemize}
        \item \begin{highlightbox}[leveltwo] Contrato de administración delegada: Contratista se encarga de la administración de la obra, y el mandante paga los costos de la obra y un porcentaje adicional por la administración (honorarios). El riesgo es del mandante. Se recomienda como opción de emergencia y sin competencia, cuando se tiene completo el proyecto y se debe cumplir en un plazo corto. Se requiere confianza e inspecciones constantes. \end{highlightbox}
        \begin{itemize}[label={},left=2em,align=parleft]
            \item \begin{highlightbox}[levelthree] Dueño no conoce el presupuesto final. \end{highlightbox}
            \item \begin{highlightbox}[levelthree] Contratista no corre riesgo con ganancias. \end{highlightbox}
            \item \begin{highlightbox}[levelthree] Contratista puede encarecer la obra. \end{highlightbox}
            \item \begin{highlightbox}[levelthree] Si los honorarios son fijos, contratista se motiva a terminar antes. \end{highlightbox}
            \item \begin{highlightbox}[levelthree] Si honorarios tienen incentivo por horario/plazo, contratista se motiva a cumplir. \end{highlightbox}
        \end{itemize}
    \end{itemize}
    \item \begin{highlightbox}[levelone] Condiciones previas al llamado de una propuesta. \end{highlightbox}
    \begin{itemize}[label={},left=1em,align=parleft]
        \item \begin{highlightbox}[leveltwo] Mandante debe tener claro lo que se quiere construir, el costo, el financiamiento y adicionales. \end{highlightbox}
        \item \begin{highlightbox}[leveltwo] Mandante debe informar al proyectista el costo aproximado de la obra. \end{highlightbox}
        \item \begin{highlightbox}[leveltwo] Mandante debe avisar al proyectista el tipo de contrato que se concretará. \end{highlightbox}
        \item \begin{highlightbox}[leveltwo] Incluir método constructivo en el diseño del proyecto. \end{highlightbox}
        \item \begin{highlightbox}[leveltwo] Se deben elaborar las bases administrativas por las que se regirá el contrato. \end{highlightbox}
        \item \begin{highlightbox}[leveltwo] Mandante podría encargar un estudio de presupuesto e inversión oficial de la obra. Se suele saltar esto. \end{highlightbox}
        \item \begin{highlightbox}[leveltwo] Establecer clara y rígidamente el sistema de pago que se implantará y la fuente de financiamiento de la obra. \end{highlightbox}
        \item \begin{highlightbox}[leveltwo] Elaborar el proyecto a cabalidad y en lo posible concertar una o más reuniones con todos los proyectistas participantes. \end{highlightbox}
        \item \begin{highlightbox}[leveltwo] Existen propuestas públicas (todos los que cumplan los requisitos) y privadas (aquellos invitados). \end{highlightbox}
        \item \begin{highlightbox}[leveltwo] OJO: El Estado está obligado por carta fundamental a llamar licitaciones públicas en primera instancia. \end{highlightbox}
        \item \begin{highlightbox}[leveltwo] Registro y pre-clasificación de contratistas: \end{highlightbox}
        \begin{itemize}[label={},left=2em,align=parleft]
            \item \begin{highlightbox}[levelthree] Clasifican a las empresas por especialidad, tamaño, experiencia, capital, etc. \end{highlightbox}
            \item \begin{highlightbox}[levelthree] Antes del llamado de licitación, se debe preseleccionar número de proponentes y los requisitos mínimos que debe satisfacer el contratista. \end{highlightbox}
            \item \begin{highlightbox}[levelthree] A todos se les entrega la misma información, calendario estricto del proceso de licitación en lo que se refiere a retiro de bases y antecedentes, plazo para consultas, plazo para respuestas, fecha de apertura o de recepción de ofertas y fecha de adjudicación de la obra. \end{highlightbox}
            \item \begin{highlightbox}[levelthree] Establecer plazo máximo para ofertar. \end{highlightbox}
        \end{itemize}
        \item \begin{highlightbox}[leveltwo] Documentos principales de una propuesta: \end{highlightbox}
        \begin{enumerate}
            \item \begin{highlightbox}[levelfour] Instrucciones a los proponentes. \end{highlightbox}
            \item \begin{highlightbox}[levelfour] Bases generales. \end{highlightbox}
            \item \begin{highlightbox}[levelfour] Propuesta o formularios de la propuesta. \end{highlightbox}
            \item \begin{highlightbox}[levelfour] Bases especiales. \end{highlightbox}
            \item \begin{highlightbox}[levelfour] Especificaciones técnicas. \end{highlightbox}
            \item \begin{highlightbox}[levelfour] Planos del proyecto. \end{highlightbox}
            \item \begin{highlightbox}[levelfour] Documentos de referencia. \end{highlightbox}
            \item \begin{highlightbox}[levelfour] Serie de preguntas y respuestas. \end{highlightbox}
            \item \begin{highlightbox}[levelfour] Apéndices. \end{highlightbox}
            \item \begin{highlightbox}[levelfour] Antecedentes técnicos complementarios sobre el terreno o sus accesos. \end{highlightbox}
        \end{enumerate}
        \item \begin{highlightbox}[leveltwo] Evaluación y adjudicación de una propuesta. \end{highlightbox}
        \begin{itemize}[label={},left=2em,align=parleft]
            \item \begin{highlightbox}[levelthree] Propuestas son recibidas y abiertas por una comisión designada por el propietario, durante una reunión donde se leen algunos datos relevantes y se registran en un acta de apertura. Este proceso puede realizarse de manera electrónica, garantizando transparencia para todos los oferentes y el público. Un ejemplo de esto es el portal mercadopúblico.cl. Se emiten dos actas: una de apertura y otra de evaluación de ofertas, que se crean tras la apertura técnica y económica. Durante la evaluación, se realiza un análisis comparativo de las ofertas técnicas y económicas. \end{highlightbox}
            \item \begin{highlightbox}[levelthree] La nota final de evaluación técnica se calcula de la siguiente manera: \end{highlightbox}
            \begin{equation}
                NFt = \sum_{i=1}^{n} (X_i \times Y_i)
            \end{equation}
            Con:
            \begin{equation}
                \sum_{i=1}^{n} Y_i = 1 
            \end{equation}
            Donde:
            \begin{itemize}[label={},left=3em,align=parleft]
                \item \begin{highlightbox}[levelfour] $X_1$ = Experiencia y antecedentes de la empresa \end{highlightbox}
                \item \begin{highlightbox}[levelfour] $X_2$ = Tipo de organización y metodología que se ofrecen \end{highlightbox}
                \item \begin{highlightbox}[levelfour] $X_3$ = Equipo de trabajo ofrecido \end{highlightbox}
                \item \begin{highlightbox}[levelfour] $X_4$ = Seriedad \end{highlightbox}
                \item \begin{highlightbox}[levelfour] $X_5$ = Capacidad económica \end{highlightbox}
                \item \begin{highlightbox}[levelfour] $X_6$ = Capacidad técnica \end{highlightbox}
            \end{itemize}
            \item \begin{highlightbox}[levelthree] Luego se hace la evaluación económica sobre la base del valor de la oferta (NFe). Se obtiene la nota final (NF) usando la ponderación para cada evaluación. \end{highlightbox}
            \begin{equation}
                NF = NFt \times P_t + NFe \times P_e
            \end{equation}
        \end{itemize}
    \end{itemize}
\end{itemize}

\end{document}




